\chapter{Relevant Publications}

Much of the work presented in this thesis has been previously published in peer-reviewed venues.
However, in various places the experiments have been improved and results have been subsequently updated.
In this appendix, we reference these publications in approximate order of appearance in this thesis and outline how the work presented here differs.

\begin{description}
\item[Extracting Ground-Truth Information from MIDI Files]$\;$\\ \Cref{ch:midi} and \cref{sec:reliability} appear together in a paper to appear at the 17th International Society for Music Information Retrieval conference \cite{raffel2016extracting}.
They are presented in this thesis largely unchanged, except for some editing to make them better fit into the storyline.

\item[Optimizing DTW-Based Audio-to-MIDI Alignment and Matching]$\;$\\ Much of \cref{ch:dtw} was published in a paper at the 41st International Conference on Acoustics, Speech, and Signal Processing \cite{raffel2016optimizing}.
For better explanation of the results, we added \cref{fig:corruption,fig:warping,fig:correction_error,fig:correlation} and \cref{tab:best_systems,tab:ratings}.
We also added the normalized DTW distance re-scaling step defined in \cref{eq:score_normalization}, which makes it more readily interpretable as a confidence score.

\item[Large-Scale Content-Based Matching of MIDI and Audio Files]$\;$\\ The approach of \cref{ch:dhs} was originally proposed in a paper at the 16th International Society for Music Information Retrieval Conference \cite{raffel2015large}.
However, it is presented in this thesis with many improvements.
First, in \cite{raffel2015large}, beat-synchronous spectrograms are used (i.e.\ the beats of the music signal are estimated and frames in the spectrogram within each beat are aggregated).
This caused issues when the beats of a given MIDI file were estimated differently than its correct match.
We therefore forewent beat synchronization and instead used a finer, fixed timescale.
This prompted the use of three maxpooling-by-2 layers, resulting in downsampling in time by 8, rather than the two maxpooling layers used in \cite{raffel2015large}.
Second, we utilized the ``gold standard` system of \cref{sec:goldstandard} to create the training data rather than the alignment scheme proposed in \cite{raffel2015large} for consistency and due to its better performance.
Third, we used 32-bit binary vectors rather than 16-bit, which potentially gives them more expressive power.
It also resulted in the use of the POPCNT instruction being more efficient than the pre-computed table look-up used in \cite{raffel2015large}.
Finally, we experimented with a wider variety of system designs, all of which are documented in \cref{sec:hashing_system}.
This included the output $L^2$ penalty, $2 \times 1$ pooling, and the larger number of small $3 \times 3$ convolutional layers, rather than the $5 \times 12$ input filter size.
Taken together, these changes resulted in dramatic increases in precision; specifically, we were improved our high-confidence pruning threshold from 10,000 to 250.
In addition, for better framing of our results, we added the baseline learning-free approach described in \cref{sec:baseline}.

\item[Accelerating Multimodal Sequence Retrieval with Convolutional Networks]$\;$\\ In \cref{ch:dhs}, we also include the ``multimodal'' variant of the experiment which was first explored in a paper at the 2015 Workshop on Multi-Modal Machine Learning \cite{raffel2015accelerating}.
We improved upon this approach in the same way as that of \cite{raffel2015large}, which also resulted in substantial performance gains.

\item[Pruning Subsequence Search with Attention-Based Embedding]$\;$\\ Our feed-forward attention-based sequence embedding model was first proposed in a paper at the 41st International Conference on Acoustics, Speech, and Signal Processing \cite{raffel2016pruning}.
As presented in \cref{ch:pse}, it is largely unchanged, except that we present a broader overview of previous work on attention and embedding and remove sections which are redundant with \cref{ch:dhs}.
We also re-performed the hyperparameter optimization and experimented with various other system design choices as outlined in \cref{sec:pse_specifics}.
\end{description}
