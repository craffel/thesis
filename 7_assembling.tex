\chapter{Assembling a Collection of MIDI Files Matched to the Million Song Dataset}
\chaptermark{Assembling a Collection of MIDI Files Matched to the MSD}
\label{ch:assembling}

Combining the results of \cref{ch:dtw,ch:dhs,ch:pse}, we now have all the ingredients required to efficiently match MIDI files to the Million Song Dataset.
Taken together, these three techniques present a cascade of matching methods, each of which achieves more precision at higher computational cost than the last.
This allows us to quickly discard the vast majority of the dataset before utilizing the expensive, but accurate, DTW operation of \cref{ch:dtw}.
While our main application in this thesis is MIDI-to-audio matching, all of these techniques are learning-based and data-adaptive, so we are optimistic they will be effective in other domains.

In this chapter, we finally utilize these techniques to match our collection of 178,561 MIDI files to the MSD.
In the following section, we discuss the practical implementation of this process, perform a small experiment to verify its performance, and give an overview of the results.
In \cref{sec:deriving}, we quantify the reliability of ground-truth annotations for content-based MIR extracted from this data and discuss further potential applications.
Finally, in \cref{sec:ideas} we wrap up with ideas for additional applications and improvements for our approach.

\section{Matching MIDI Files}

As an overview, our method for matching MIDI files to the MSD will proceed as follows:
Given a query MIDI file, we first synthesize it using \prettymidi{}'s \texttt{fluidsynth} method.
Then, we compute the log-magnitude, $L^2$-normalized, constant-Q spectrogram representation utilized throughout this thesis.
Using the pre-trained models from \cref{ch:dtw,ch:pse}, we compute the MIDI file's downsampled hash sequence and embedding.
We then compute the pairwise distance between the resulting embedding and all pre-computed embeddings of entries in the MSD, rank the entries according to their distance, and discard all but the top $N_{PSE}$.
We perform a similar pruning with the downsampled hash sequences, computing the DTW distance between the query and top $N_{PSE}$ MSD entries, ranking by distance, and discarding all but the top $N_{DHS}$.
Finally, we align the MIDI file to the remaining $N_{DHS}$ entries utilizing the system of \cref{sec:goldstandard}.

Before proceeding, we must first decide the pruning thresholds $N_{PSE}$ and $N_{DHS}$.
Choosing these thresholds directly trades off precision and computation, because setting larger thresholds will decrease the likelihood of discarding a valid match but will result in more comparisons being made downstream.
Based on the results of \cref{sec:ranking,sec:pse_results}, we chose $N_{PSE} = 100,000$ and $N_{DHS} = 250$.
This corresponds to pruning 90\% of the MSD in the first step, then pruning 99.75\% of the remaining entries, so that in the end the expensive DTW comparison is only performed on 0.025\% of the MSD.
According to our test-set results, this results in a false-reject rate of approximately 3.5\% for the embedded distance pruning and xxx\% for the downsampled hash sequences.
Assuming false rejects occur in each step independently, we can expect a combined false reject rate of up to xxx\%.

\subsection{Shortcuts}

Overall, we will follow the matching processes of \cref{sec:goldstandard,sec:hashing_system,sec:pse_specifics}, except for the following two changes made for efficiency:

First, we utilize an exact pruning method when comparing downsampled hash sequences.
Assuming we are computing the DTW distance as defined in \cref{sec:dtw} between two sequences, one of length $M$ and one of length $N$ with $M \le N$, then the normalized DTW distance will never be smaller than $T$ if there are not at least $M$ pairwise distances less than $T$.
Given this, while computing the pairwise distance matrix for downsampled hash sequences, we can (with a minor performance penalty) keep track of the number of entries which are smaller than $T$, where $T$ is set to the smallest normalized DTW distance which has been encountered so far.
If fewer than $M$ entries are less than $T$, we can skip computing the DTW alignment path because we can guarantee that the resulting normalized DTW distance will not be smaller than $T$.
In practice, this avoids the expensive operation of computing the DTW distance for many candidates whose distance matrices dictate that they cannot be the correct match.

Second, instead of utilizing the median pairwise distance for the CQT-to-CQT DTW additive penalty as suggested in \cref{sec:goldstandard}, we will use the mean distance.
This choice is largely made for performance reasons; computing the median of the $MN$ distances is at least $\mathcal{O}( NM\log(NM) )$ complexity whereas computing the mean is always $\mathcal{O}(NM)$.
We justify this decision based on the discussion in \cref{sec:optimizing} which suggests that small changes to the additive penalty will not affect alignment quality.
In practice, we found the mean and median pairwise distance to be extremely close, so this change likely has a negligible effect on the results aside from a small speed-up.

\subsection{Measuring Performance}

Prior to performing the matching of our full collection of MIDI files to the MSD, it is useful to get an idea of the expected accuracy of the process described above.
We therefore utilized this approach on the same evaluation task used in \cref{ch:dhs,ch:pse}, i.e.\ measuring the accuracy of matching our test set of known-correct MIDI-MSD pairs.
Specifically, we utilized the methods of \cref{ch:pse,ch:dhs} to prune all but 100,000 and 250 entries respectively and then computed the match confidence score for the remaining 250 entries using the method of \cref{sec:goldstandard}.
For evaluation, we then ranked the remaining 250 entries according to their resulting confidence scores.
If the correct match in the MSD was pruned incorrectly, we assigned a rank 1,000,000, because effectively this results in the correct matching being unrecoverable by ouer system.
For full details of this evaluation task, refer to \cref{sec:dhs_experiment}.

\begin{figure}
  \includegraphics[width=\textwidth]{figures/7-ranks.pdf}
  \caption[Percentage of the test set below a given rank]{Percentage of MIDI files in the test set whose correct match in the MSD ranked better than a given threshold, using the ``DHS'' method of \cref{ch:dhs}, ``PSE'' of \cref{ch:pse}, and their combination with the method of \cref{sec:goldstandard}.}
  \label{fig:combined_ranks}
\end{figure}

The proportion of pairs for which the correct entry matched above a given rank threshold is shown in \cref{fig:combined_ranks}.
For comparison, we also plot the curves obtained using the approaches of \cref{ch:dhs,ch:pse}.
Using the combined method, the correct match in the MSD had the highest confidence score 78.2\% of the time, and for 94.6\% of the test set the correct match ranked in the top 5.
Beyond this point, the performance is effectively constant - this is because for 94.7\% of the test set, the pruning steps discarded the correct match.
This is within our expected upper limit for the false reject rate of xxx\%.
That our high-precision approach only ranked the correct match first 78.2\% of the time is also expected because the MSD contains many duplicates and alternate versions of songs \cite{bertin2012large}.
It's also important to note that these figures do not necessarily reflect real-world performance because, due to the way it was constructed, our test set consists only of ground-truth matches which could successfully be recovered by our DTW-based system of \cref{sec:goldstandard}.
Nevertheless, it does demonstrate the extent to which our combination of pruning approaches causes false rejects.

Of course, our approach's precision is only beneficial if it is also efficient.
On average, matching a MIDI file to the MSD using this cascade of techniques takes xxx.
Of this, on average xxx is spent pruning all but 100,000 entries using the embedding approach, xxx is spent pruning down to 250 entries using the downsampled hash sequence approach, and xxx is spent computing confidence scores for the remaining 250 entries.
The remaining xxx is the time required to compute all of the necessary representations for the MIDI file prior to matching.
Compared to the approximately 1.5 days it would take to match a single MIDI file to the MSD using the approach from \cref{sec:goldstandard} on its own, we consider the 5.4\% false reject rate a small price to pay for this increase in efficiency.
Of course, we could increase precision substantially by raising the pruning thresholds in exchange for a drop in efficiency, but we consider this to be a good trade-off.

\subsection{The Lakh MIDI Dataset}

Now that we have empirically validated our approach, we proceed to matching our collection of 178,561 MIDI files to the MSD.
To do so, we simply followed the approach of the previous section exactly, except that we matched our entire collection against the MSD instead of just our test set.
Based on the fact that matching a single file takes about xxx on average, performing this matching for all 178,561 MIDI files takes about xxx of compute time.
To speed things up, we ran this process in parallel across 32 processors, resulting in a total run time of about xxx.
While performing the match, we discovered xxx MIDI files which were corrupt or otherwise unparsable by \prettymidi{}.
In addition, in order to avoid spurious matches and incomplete transcriptions, we discarded all MIDI files which were less than 30s in length.
In the end, we therefore matched a total of xxx MIDI files to the MSD.

\begin{figure}
  \includegraphics[width=\textwidth]{figures/7-matches.pdf}
  \caption[Number of MIDI files with matches above a given confidence]{Number of MIDI files which had at least one entry in the MSD whose DTW confidence score was above a given threshold.
 For example, about 50,000 MIDI files had at least one MSD entry whose match confidence score was above 0.5.}
  \label{fig:match_confidences}
\end{figure}

An informative way to report the extent to which we successfully found matches in the MSD is to plot the number of MIDI files in our collection which had at least one MSD entry against which the DTW confidecne score was below a threshold.
We plot this curve in \cref{fig:match_confidences}.
Choosing 0.5 as the threshold above which a confidence score denotes a match (as we did in \cref{sec:training_data} based on the results of \cref{sec:qualitative}), we found 50,513 MIDI files which had at least one match in the MSD, and a total of 138,026 MIDI-to-MSD pairings.
Many MIDI files were matched to multiple MSD entries due to duplicates, covers, and alternate versions; for example, one MIDI transcription of ``Imagine'' by John Lennon was matched to six original renditions, ten covers, one hip-hop song which sampled it (``Image'' by Nas), and one song which used an identical chord progression (``Bringin' Down Dinner'' by Neil Young).
On the other hand, many MSD entries were matched to multiple MIDI files; the most commonly-matched MSD entry was xxx which xxx MIDI files matched to.

% TODO: Note about decoys?

Of course, the main purpose of creating this dataset is to facilitate MIR research by sharing it.
Following the example of the Million Song Dataset, we name this collection the ``lakh MIDI dataset'' (LMD), because depending on how you count we are providing about 100,000 MIDI files.
Towards this end, we make it available in the following formats:
First, the entire collection of 178,561 unique MIDI files, unmodified, along with all potential metadata available from their filenames (referred to as ``LMD-full'').
Second, a mapping between each of these MIDI files to MSD entries they were successfully matched to, along with the resulting DTW confidence score (referred to as ``LMD-matched'').
Third, for each successful MIDI-MSD pair, a pre-aligned MIDI file utilizing our ``gold standard'' system from \cref{sec:goldstandard} (referred to as ``LMD-aligned'').
By distributing it in these varying levels of granularity, we also allow future researchers to improve upon our matching and alignment processes.
In addition, we note that should a new collection of MIDI files become available, our process could easily be re-run and the resulting datasets expanded.
Alternatively, we could also easily leverage other digital score formats for which many transcriptions are available such as MusicXML \cite{good2001musicxml} or GuitarPro.\footnote{\url{http://www.guitar-pro.com}}

\section{Deriving Ground-Truth Information}
\label{sec:deriving}

Prompted by the availability of this new dataset, in this section we will investigate its applicability to different tasks.
On its own, the ``full'' dataset itself is most applicable to corpus studies of music structure and patterns which are ``metadata-agnostic''.
For example, \cite{mauch2012corpus} analyzed a smaller corpus of MIDI files to determine common drum patterns.
Similar analyses could be performed on other aspects of the transcriptions, such as common chord progressions and melodic patterns.
While LMD-full cannot be used on its own to obtain ground truth for audio content-based MIR tasks, MIDI files themselves provide a useful mechanism of obtaining synthetic data, as demonstrated in \cref{ch:dtw} and utilized for onset detection in \cite{bello2005tutorial}.
In addition, there are ways in which non-matched, non-aligned transcriptions can be used to benefit content-based MIR tasks.
For example, in automatic transcription systems it is common to smooth predictions with a ``musical language model'' which can predict the probability of a given candidate transcription \cite{poliner2007discriminative,sigtia2015end}.
Such language models only require a large amount of transcriptions, not transcriptions matched to corresponding audio recordings.
Of course, there is no guarantee that all of the MIDI files in LMD-full are actually full transcriptions because MIDI files are also sometimes used to store short snippets of music or even example riffs and drum beats for composition.

In contrast, we can say with high confidence that files in LMD-matched are transcriptions of songs.
This greatly augments the information available for those MSD entries which have been matched to transcriptions.
For example, in the past large-scale musicological studies have been undertaken using estimated pitch features \cite{serra2012measuring,bertin2010clustering}.
These results are potentially confounded by the fact that these features only represent a high-level, noisy, and potentially error-prone approximation of the actual transcription.
We expect that similar corpus studies utilizing the matched transcriptions will produce more reliable results.
The transcriptions can also provide a stronger signal for query-by-humming retrieval systems.
On the other hand, by being matched to the MSD, the MIDI files themselves are augmented with a great deal of additional metadata, including the song's release year, lyrics, user listening preferencs \cite{jansson2015this}, and human-annotated tags and genre labels \cite{schreiber2015improving}.

Finally, based on the discussion in \cref{sec:information}, the LMD-aligned dataset can potentially be used to produce ground-truth annotations for audio content-based MIR.
For example, beat annotations for a given 7digital preview recording could be obtained by utilizing the time signature and tempo change events in the corresponding aligned MIDI file.
However, the validity of a given annotation will depend on both the transcription quality as well as the success of the alignment process.
Ideally, our alignment confidence score will reflect whether or not a given annotation is to be trusted.
To investigate further, we now focus on measuring the reliability of ground-truth data extracted from aligned MIDI transcriptions.

\subsection{Comparison to Hand-Annotated Data}
\label{sec:reliability}

A straightforward way to evaluate the quality of MIDI-derived annotations is to compare them with hand-made annotations for the same songs.
Given a MIDI transcription and human-generated ground truth data, we can extract corresponding information from the MIDI file and compare using the evaluation metrics employed in the Music Information Retrieval Evaluation eXchange (MIREX) \cite{downie2008music}.
We therefore leveraged the Isophonics Beatles annotations \cite{mauch2009omras2} as a source of ground truth to compare against MIDI-derived information.  MIDI transcriptions of these songs are readily available due to The Beatles' popularity.

Our choice in tasks depends on the overlap in sources of information in the Isophonics annotations and MIDI files.
Isophonics includes beat times, song-level key information, chord changes, and structural segmentation.
As noted in Section \ref{sec:information}, beat times and key changes may be included in MIDI files but there is no standard way to include chord change or structural segmentation information.
We therefore performed experiments to evaluate the quality of key labels and beat times available in MIDI files.
Fortuitously, these two experiments give us an insight into both song-level timing-agnostic information (key) and alignment-dependent timing-critical information (beat).
To carry out these experiments, we first manually identified 545 MIDI files from LMD-full\footnote{The Million Song Dataset does not contain any Beatles songs, so we were unable to source these files from LMD-matched or LMD-aligned.} which had filenames indicating that they were transcriptions of one of the 179 songs in the Isophonics Beatles collection; we found MIDI transcriptions for all but 11.
The median number of MIDI transcriptions per song was 2; the song ``Eleanor Rigby'' had the most, with 14 unique transcriptions.

\subsection{Key Experiment}

In our first experiment, we evaluated the reliability of key change events in MIDI files.
We followed the MIREX methodology for comparing keys \cite{ehmann2016mirex}, which proceeds as follows:
Each song may only have a single key.
All keys must be either major or minor, e.g.\ ``C\# Major'' and ``E minor'' are allowed but ``D Mixolydian'' is not.
An estimated key is given a score of 1.0 when it exactly matches a ground truth key, 0.5 when it is a perfect fifth above the ground truth key, 0.3 when it is a relative major or minor, 0.2 when it is a parallel major or minor, and 0.0 otherwise.
We utilized the evaluation library \texttt{mir\char`_eval} \cite{raffel2014mir_eval} to compute this score.

The Isophonics annotations mostly follow this format, except that 21 songs contained multiple key annotations and 7 others contained non-major/minor keys.
To simplify evaluation, we discarded these songs, leaving 151 ground truth key annotations.
Of our 545 Beatles MIDIs, 221 had no key change event and 5 had more than one, which we also omitted from evaluation.
This left 223 MIDI files for which we extracted key annotations and compared them to valid Isophonics annotations.
Because of the preponderance of C major key change events noted in Section \ref{sec:key}, we also evaluated MIDI-derived C Major and non-C major instances separately to see whether they were less reliable.

As a baseline, we also extracted keys using the QM Vamp Key Detector plugin \cite{cannam2015mirex} whose underlying algorithm is based on \cite{noland2007signal} which finds the key profile best correlated with the chromagram of a given song.
This plugin achieved the highest score in MIREX 2013, and has been the only key detection algorithm submitted in 2014 and 2015.
This gives us a reasonable expectation for a good audio content-based key estimator; to determine the extent to which human annotators agree on ground truth key labels, we also collected key labels for Beatles' songs from \texttt{whatkeyisitin.com}.
As with the Isophonics key annotations, some songs had multiple and/or modal key labels; we discarded these and ended up with 145 labels for songs in the Isophonics dataset.

\begin{table}
\begin{center}
\begin{tabular}{lrr}
  \toprule
  Source                     & Score & Comparisons \\
  \midrule
  MIDI, all keys             & 0.400 &         223 \\
  MIDI, C major only         & 0.167 &         146 \\
  MIDI, non-C major          & 0.842 &          77 \\
  QM Key Detector            & 0.687 &         151 \\
  \texttt{whatkeyisitin.com} & 0.857 &         145 \\
  \bottomrule
\end{tabular}
\end{center}
  \caption{Mean scores achieved, and the number of comparisons made, by different datasets compared to Isophonics Beatles key annotations.}
  \label{tab:key}
\end{table}

The mean scores resulting from comparing each dataset to the Isophonics annotations can be seen in Table \ref{tab:key}.
At first glance, the mean score of $0.4$ achieved by MIDI key change messages is discouraging.
However, by omitting all MIDI files with C major key events (which achieved a mean score of $0.167$), the mean score jumps to $0.842$.
This is comparable to the human baseline, and is substantially higher than the algorithmically estimated score.
We therefore propose that \textit{non-C major} MIDI key change events are as reliable as hand-annotated labels, but that C major key annotations in MIDI files are effectively useless.

\subsection{Beat Experiment}

Utilizing many of the sources of information in MIDI files in audio content-based MIR tasks depends on the precise alignment of a given MIDI file to an audio recording of a performance of the same song.
We therefore performed an additional experiment to evaluate the quality of MIDI-derived beat annotations, which are evaluated on the scale of tens of milliseconds.
Producing valid beat annotations from a MIDI file requires not only that the file's timing information is correct, but also that it has been aligned with high precision.
To align our Beatles MIDI files to corresponding audio recordings, we used the scheme from \cref{sec:goldstandard} exactly, except that we computed spectrograms with a hop size of 23 milliseconds rather than 46 milliseconds.
This finer timescale is more appropriate for the beat evaluation metrics we will use, which have tolerances measured in tens of milliseconds.

We used \prettymidi{}'s \texttt{get\char`_beats} method to extract beat times from our 545 Beatles MIDI files, and adjusted each beat's timing according to the MIDI file's alignment to corresponding audio recordings.
For evaluation, we used the {\em F-measure}, {\em Any Metric Level Total}, and {\em Information Gain} metrics described in \cite{davies2009evaluation}, as implemented in \texttt{mir\char`_eval}.
As a baseline, we also computed beat locations using the \texttt{DBNBeatTracker} from the \texttt{madmom} software library,\footnote{\texttt{https://github.com/CPJKU/madmom}} which is based on the algorithm from \cite{bock2014multi}.
This represents a state-of-the-art general-purpose beat tracker which, on the Beatles data, can reliably produce high-quality annotations.
If MIDI-derived beat annotations are to be taken as ground truth, they must achieve scores similar to or higher than the \texttt{DBNBeatTracker}.

\begin{figure}
    \centering
    \includegraphics[width=\textwidth]{figures/7-beat_scores.pdf}
    \caption[Beat evaluation and alignment confidence scores]{Beat evaluation metric scores (compared to Isophonics beat annotations) and alignment confidence scores achieved by different audio-to-MIDI alignments of Beatles MIDI files, with each shown as a blue dot.
Mean scores for each metric achieved by the \texttt{DBNBeatTracker} \cite{bock2014multi} are shown as dashed lines.}
    \label{fig:beat}
\end{figure}

We visualize the resulting scores in Figure \ref{fig:beat}.
Because we don't expect beats to be extracted accurately from MIDI files that are poor transcriptions or when alignment failed, we plotted each MIDI file as a single point whose x coordinate corresponds to the MIDI alignment confidence score and whose y coordinate is the resulting evaluation metric score achieved.
Ideally, all points in these plots would be clustered in the bottom left (corresponding to failed alignments with low confidence scores) or top right (corresponding to a successful alignment and beat annotation extraction with a high confidence score).
For reference, we plot the mean score achieved by the \texttt{DBNBeatTracker} as dotted lines for each metric.
From these plots, we can see that in many cases, MIDI-derived annotations achieve near-perfect scores, particularly for the {\em F-Measure} and {\em Any Metric Level Total} metrics.
However, there is no reliable correspondence between high confidence scores and high evaluation metric scores.
For example, while it appears that a prerequisite for an accurate MIDI-derived beat annotation is a confidence score above $.5$, there are many MIDI files which had high confidence scores but low metric scores (appearing in the bottom-right corner of the plots in Figure \ref{fig:beat}).

We found that this undesirable behavior was primarily caused by a few issues:
First, it is common that the alignment system would produce alignments which were slightly ``sloppy'', i.e.\ were off by one or two frames (corresponding to 23 milliseconds each) in places.
This had less of an effect on the {\em F-measure} and {\em Any Metric Level Total} metrics, which are invariant to small temporal errors up to a certain threshold, but deflated the {\em Information Gain} scores because this metric rewards consistency even for fine-timing errors.
Second, many MIDI files had tempos which were at a different metric level than the annotations (e.g.\ double, half, or a third of the tempo).
This affected the {\em Any Metric Level Total} scores the least because it is invariant to these issues, except for the handful of files which were transcribed at a third of the tempo.
Finally, we found that the confidence score produced by the alignment system is most reliable at producing a low score in the event of a total failure (indicated by points in the bottom left of the plots in Figure \ref{fig:beat}), but was otherwise insensitive to the more minor issues that can cause beat evaluation metrics to produce low scores.

\subsection{Motivating Future Work}

These results suggest that while MIDI files have the potential to be valuable sources of ground truth information, their usage may come with a variety of caveats.
While in many cases (discussed above in \cref{sec:deriving}) MIDI files are useful on their own, we believe that developing better methods to leverage all of the information present in them is a tantalizing avenue for obtaining reliable ground truth data for music information retrieval.
For example, while C-major key annotations cannot be trusted, developing a highly reliable C-major vs. non-C-major classification algorithm for symbolic data (which would ostensibly be much more tractable than creating a perfect general-purpose audio content-based key estimation algorithm) would enable the reliable usage of all key change messages in MIDI files.
Further work into robust audio-to-MIDI alignment is also warranted in order to leverage timing-critical information, as is the neglected problem of alignment confidence score reporting.
Novel questions such as determining which MIDI instrument corresponds to the vocal or melody instrument in a given transcription, or determining whether all instruments have been transcribed, would also facilitate the use of MIDI files as ground truth transcriptions.
Fortunately, all of these tasks are made easier by the fact that MIDI files are specified in a format from which it is straightforward to extract pitch information.
We hope that the release of the Lakh MIDI Dataset prompts further research into these problems, so that all of the information sources in MIDI files may be fully utilized.

\section{Ideas and Improvements}
\label{sec:ideas}

Beyond the potential uses for our new dataset outlined above, we also are interested in extending the techniques and applications we proposed in this thesis.
We are most interested in fusing the different approaches we explored.
For example, we believe a representation learning approach like the one used in \cref{ch:dhs} could make DTW more effective in the final audio-to-MIDI alignment step.
This is prompted by our observation in \cref{sec:discussion} that the best representation for alignment is not necessarily the best for confidence reporting.
The success of representation learning in \cref{ch:dhs} in creating a more efficient and similarity-preserving representation for DTW suggests this approach may also be beneficial in the final high-precision alignment.

An issue with the training process utilized in \cref{ch:dhs} is that it requires training data in the form of pre-aligned sequences.
More broadly, this method of training does not optimize DTW performance directly, but instead optimizes the representation without considering the effect on the DTW alignment or distance.
A more principled approach, then, would be to learn both the representation and the alignment procedure together as part of a larger system.
Recently, a variety of approaches have been proposed to learn alignment as part of either the model's loss function (e.g.\ the Connectionist Temporal Classification loss \cite{graves2006connectionist}) or model architecture (e.g.\ attention models \cite{bahdanau2014neural} or pointer networks \cite{vinyals2015pointer}).
We are interested in applying these methods to jointly learning a representation and a procedure for alignment.

In this thesis we developed a cascade of methods of varying precision and efficiency for large-scale sequence similarity searches.
Each step was developed independently, which may be suboptimal.
A better approach may be to simultaneously learn more and more efficient representations by repeatedly downsampling to coarser timescales, as is done in FastDTW \cite{salvador2007toward}.
However, rather than downsampling by simply computing the average of non-overlapping groups of subsequent feature elements as in \cite{salvador2007toward}, we could utilize the same learning procedure of \cref{ch:dhs} to produce efficient data-driven approximations at each timescale.
We expect that this hierarchical approach would allow for even more efficient pruning.

Finally, we are interested in testing these techniques in domains outside of music.
We suspect they will be beneficial in large-scale sequence comparison problems where sequences are oversampled, consist of high-dimensional feature vectors, and/or have elements which are not directly comparable by Euclidean distance.
We hope the results and methods proposed in this thesis prompt novel research into learning-based methods for efficiently and effectively comparing sequences.
