\chapter{Tools} \label{ch:tools}

Throughout this thesis, we will use a common collection of tools to help solve problems of interest.
Broadly speaking, we will make use of techniques from the fields of machine learning and signal processing.
In this chapter, we give a high-level overview of these fields and a more specific definition of the various techniques we will utilize.

\section{Machine Learning}

Machine learning is broadly defined as a suite of methods for enabling computers to carry out particular tasks without being explicitly programmed to do so.
Such methods make use of a collection of data, from which patterns or the desired behaviors are automatically determined.
In this sense, machine learning techniques allow computers to ``learn'' the correct procedure based on the data provided.
In addition to requiring data, machine learning algorithms also require an objective which quantifies the extent to which they are successfully carrying out a task.
This objective function allows them to be optimized in order to maximize their performance.

Traditionally, machine learning techniques are divided into three types: Supervised, unsupervised, and reinforcement learning.
Supervised learning requires a collection of training data that specifies both the input to the algorithm and the desired output.
Unsupervised learning, on the other hand, does not utilize target output data, and thus is limited to finding structure in the input data.
Finally, reinforcement learning refers to the broader problem of determining a policy to respond to input data in a dynamic ``environment'' based on a potentially rare signal which tells the algorithm whether it is successful or not.
All of the problems in this thesis fall into the category of supervised learing, so we will not discuss unsupervised or reinforcement learning further.

\subsection{Supervised Learning}

\subsubsection{Sequential Data}

\subsection{Neural Networks}

\subsubsection{Recurrent Networks}

\subsubsection{Convolutional Networks}

\subsubsection{Attention}

\subsection{Stochastic Optimization}

\subsection{Bayesian Optimization}

\section{Signal Processing}

\subsection{Audio Signals and Psychoacoustics}

proposal1

\subsection{Time-Frequency Analysis}

\subsection{Dynamic Time Warping}

ismir2015large2, icassp2016optimizing2

\subsubsection{Pruning Methods}

icassp2016pruning1
