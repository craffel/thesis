\chapter{Software Prepared}

Over the course of my PhD, I wrote over 50,000 lines of Python code, all of which are open-source and publicly available online.\footnote{\url{http://github.com/craffel}}
This includes the requisite code for running the experiments in this thesis and my other research projects, in addition to contributions to more generally-applicable software libraries.
This is arguably the main product of my PhD work, so in this appendix I give an overview of the software utilized in this thesis which I developed or helped develop.

\section{\texttt{midi-dataset}}

The repository \texttt{midi-dataset}\footnote{\url{http://github.com/craffel/midi-dataset}} is a collection of code which implements all of the experiments in \cref{ch:dhs,ch:pse,ch:assembling}.
Specifically, it includes the following scripts, in order of their intended use:

\begin{description}
\item[\texttt{create\char`_msd\char`_cqts.py}] Pre-computes constant-Q spectrograms for all of the 7digital preview clips \cite{schindler2012facilitating} from the entire Million Song Dataset \cite{bertin2011million} in the format described in \cref{sec:hashing_system} and used throughout this thesis.
\item[\texttt{create\char`_whoosh\char`_indices.py}] Builds search indices for the Python library \texttt{whoosh} of all datasets used in these experiments (the MSD, the ``clean MIDI subset'', uspop2002, CAL10k, and CAL500) to allow for the fuzzy text-based matching utilized in \cref{sec:training_data}.
\item[\texttt{text\char`_match\char`_datasets.py}] Runs fuzzy text matching of the clean MIDI subset against each of the aforementioned audio datasets using \texttt{whoosh} in order to obtain candidate matches.
\item[\texttt{align\char`_text\char`_matches.py}] Runs the audio-to-MIDI alignment process described in \cref{sec:goldstandard} on all of the text matches to find true ground-truth MIDI-audio matches as described in \cref{sec:training_data}.
\item[\texttt{split\char`_training\char`_data.py}] Creates the fixed train/validation/development/test split {\em by song} of the ground-truth MIDI-audio matches.
\item[\texttt{experiments}] A collection of experiments for evaluating all of the matching techniques we proposed and described.
Each experiment optionally includes a script \texttt{parameter\char`_search.py} which optimizes the parameters of the system using the training and validation sets.
All experiments include scripts \texttt{precompute.py} and \texttt{match\char`_msd.py} which respectively pre-compute all necessary data for the experiment (for example, fixed-length embeddings of the entire MSD and clean MIDI test set) and evaluate the matching on the test set.
Experiments included are:
\begin{description}
\item[\texttt{dhs}] for the learning-based method for creating downsampled hash sequences described in \cref{sec:hashing_system}.
\item[\texttt{dhs\char`_piano}] for the corresponding experiment using piano rolls as a MIDI representation rather than synthesized spectrograms, described in \cref{sec:multimodal}.
\item[\texttt{tpaa}] for the ``thresholded piecewise aggregate approximation'' baseline described in \cref{sec:baseline}.
\item[\texttt{pse}] for the learning-based method for producing similarity-preserving embeddings of variable-length sequences using a feed-forward network with attention described in \cref{sec:pse_experiment}.
\item[\texttt{stats}] for the feature statistics-based embedding baseline described in \cref{sec:embedding_baseline}.
\item[\texttt{combined}] for the experiment from \cref{ch:assembling} evaluating the performance of the combination of the above learning-based methods and the DTW-based matching system described in \cref{sec:goldstandard}.
\end{description}
\item[\texttt{match.py}] Matches one or more MIDI files to the million song dataset quickly and accurately using the cascade of methods proposed in this thesis, as described in \cref{ch:assembling}.
\end{description}

Also included are \texttt{experiment\char`_utils.py}, \texttt{feature\char`_extraction.py}, and \texttt{whoosh\char`_search.py} which contain shared helper functions for the experiments, extracting features, and text-based matching respectively.

\subsection{\texttt{dhs}}

To facilitate its usage in contexts outside of the experiments in this thesis, the code for the learning-based downsampled hash sequence technique proposed in \cref{ch:dhs} is factored out in a separate library called \texttt{dhs}.\footnote{\url{http://github.com/craffel/dhs}}
\texttt{dhs} includes the following functionality:
\begin{description}
\item[\texttt{train.py}] Given a dataset of sequential data with known-matching pairs and an un-trained pair of neural network models, train the networks by optimizing the similarity-preserving loss \cref{eq:hashing_objective}.
\item[\texttt{match.py}] Given hash sequences represented as lists of integers, rapidly compute the pairwise distance matrix using exclusive-or and POPCNT operations and the DTW distance corresponding to the lowest-cost path through this matrix.
\item[\texttt{utils.py}] Tools for iterating over sequential data and constructing ``dissimilar'' pairs by randomly swapping matching pairs.
\end{description}
This library makes use of \texttt{theano} \cite{bergstra2010theano,bastien2012theano,al-rfou2016theano} and \texttt{lasagne} \cite{dieleman2015lasagne} to construct the loss function, automatically compute its derivative, and optimize the networks' parameters with respect to it.
It also utilizes \texttt{numba}\footnote{\url{http://numba.pydata.org}} to rapidly compute the DTW distance and \texttt{cython}\footnote{\url{http://cython.org}} to access the SSE 4.2 POPCNT instruction directly from Python code.
While thus far this library has only been used for the experiments in this thesis, we hope it will be facilitate research into other problem settings.

\subsection{\texttt{pse}}

In the same vein as \texttt{dhs}, \texttt{pse}\footnote{\url{http://github.com/craffel/pse}} is a library which covers the pairwise sequence embedding approach of \cref{ch:pse}.
In general, it has very similar functionality to the \texttt{dhs} library except that it utilizes non-aligned pairs of sequences, it expects the networks to produce a single vector for each input sequence, and it does not contain functionality for matching the resulting embeddings because this can be achieved by a simple Euclidean distance calculation.
It also includes an implementation of the feed-forward attention mechanism described in \cref{sec:ffattention}.

\section{\texttt{alignment-search}}

Our code is also available online for the experiments in \cref{ch:dtw}.\footnote{\url{http://github.com/craffel/alignment-search}}
This includes the following scripts, in order of their intended usage:
\begin{description}
\item[\texttt{create\char`_data.py}] Given a collection of MIDI files, creates ``easy'' and ``hard'' artificial synthetic corrupted versions as described in \cref{sec:synthetic}.
\item[\texttt{parameter\char`_experiment\char`_random.py} and \texttt{parameter\char`_experiment\char`_gp.py}] Optimize DTW system hyperparameters over the ``easy'' synthetic dataset, first randomly and then using Bayesian optimization, as described in \cref{sec:optimizing}.
\item[\texttt{find\char`_best\char`_aligners.py}] Finds the best-performing alignment system produced by hyperparameter optimization and all of the systems which were not significantly worse.
\item[\texttt{confidence\char`_experiment.py}] Given these high-performing systems, determine which of the confidence score reporting schemes discussed in \cref{sec:confidence} best correlates with alignment error over both the ``easy'' and ``hard' datasets.
\item[\texttt{align\char`_real\char`_world.py}] Aligns a collection of real-world data using the ``gold standard'' system of \cref{sec:goldstandard} for the qualitative evaluation in \cref{sec:qualitative}.
\end{description}

It also includes the collections of utility functions \texttt{corrupt\char`_midi.py}, \texttt{align\char`_dataset.py}, and \texttt{db\char`_utils.py} respectively for applying synthetic corruptions to MIDI files, aligning a collection of MIDI files given DTW system parameters, and for storing and loading results.

\section{\prettymidi{}}
\label{sec:pretty_midi}

Because MIDI files utilize a low-level binary protocol to represent musical scores, manipulating MIDI data directly can be cumbersome.
In addition, the timing of an event in MIDI is represented by tempo-dependent ``ticks'' relative to the previous event, making direct interpretation in terms of absolute time (in seconds) difficult.
We believe that the most intuitive representation is a hierarchical one, consisting of a list of instruments, each of which contains a sequence of events (notes, pitch bends, etc.).
This is analogous to a per-instrument piano roll, the visualization commonly used for manipulating MIDI data in digital audio workstations.

Most software libraries for dealing with MIDI files either represent MIDI data in a lower-level (directly corresponding to the bit-level representation) or a higher-level manner (only as musical features).
This can make simple manipulations and analysis either require a great deal of source code and expertise or, in the case of modifications to aspects not supported by an abstract library, impossible.
For example, in the python-midi module,\footnote{\url{http://github.com/vishnubob/python-midi}} shifting up the pitch of all notes in a MIDI file by 2 semitones takes only a few lines of code.
However, constructing a piano roll representation can take a few hundred lines of code because MIDI ticks must be converted to time in seconds, note-on events must be paired with note-offs, drum events must be ignored, and so on.

Based on these issues, we created the Python module \prettymidi{}\footnote{\url{http://github.com/craffel/pretty-midi}} for creating, manipulating and analyzing MIDI files.
It is intended to make the most common operations applied to MIDI data as straightforward and simple as possible.
The module includes functionality for parsing and writing MIDI files, creating and manipulating MIDI data, synthesis, and information extraction.

\prettymidi{} represents MIDI data as a hierarchy of classes.
At the top is the \texttt{PrettyMIDI} class, which contains global information such as tempo changes and the MIDI resolution.
It also contains a list of \texttt{Instrument} class instances.
Each \texttt{Instrument} is specified by a program number and a flag denoting whether it is a drum instrument.
\texttt{Instrument} class instances also contain three lists, one each for \texttt{Note}, \texttt{PitchBend}, and \texttt{ControlChange} class instances.
The \texttt{Note} class is a container for MIDI notes, with velocity, pitch, and start and end time attributes.
Similarly, the \texttt{PitchBend} and \texttt{ControlChange} classes simply have attributes for the bend or control change's time and value.

The top-level \texttt{PrettyMIDI} class can be instantiated with a path to an existing MIDI file, in which case the class will be populated by parsing the file.
It can also be instantiated without a pre-existing file for creating MIDI data from scratch.
For output, the \texttt{PrettyMIDI} class has a \texttt{write} function which exports its data to a valid MIDI file.

\texttt{PrettyMIDI} class instances have functions for performing analysis on the data they contain, some of which have a corresponding function in the \texttt{Instrument} class.
Some of the implemented functions include:

\begin{description}
  \item[\texttt{get\char`_tempo\char`_changes}] Returns a list of the times and tempo (in beats per minute) of all MIDI tempo change events
  \item[\texttt{estimate\char`_tempo}] Returns an empirical tempo estimate according to inner-onset intervals, as described in \cite{dixon2001automatic}.
  \item[\texttt{get\char`_beats}] Returns a list of beat locations based on the MIDI tempo change events.
  \item[\texttt{get\char`_onsets}] Returns a list of all of the onsets (start times) of each MIDI note.
  \item[\texttt{get\char`_piano\char`_roll}] Returns a piano roll matrix representation of MIDI notes, as used in \cref{sec:multimodal}.
  \item[\texttt{get\char`_chroma}] Computes chroma features (also known as pitch class profile)~\cite{fujishima1999realtime} of the MIDI data based on the piano roll representation.
\end{description}

In \prettymidi{}, MIDI data can be synthesized as audio using either the \texttt{synthesize} or \texttt{fluidsynth} methods.
\texttt{synthesize} uses a periodic function (e.g. sin) to synthesize the each note, while \texttt{fluidsynth} utilizes the \texttt{fluidsynth} program\footnote{\url{http://www.fluidsynth.org}} which performs General MIDI synthesis using a SoundFont file.
Both methods return arrays of audio samples at some specified sampling rate.

\prettymidi{} also has utility functions for converting between representations of MIDI notes (name, note number, frequency in Hz, and drum name for percussion instruments), program number and instrument name/class (according to the General MIDI standard) and pitch bend value and semitones.
Each \texttt{PrettyMIDI} class instance can also readily convert between MIDI ticks and absolute seconds.
These functions allow for semantically meaningful creation and representation of MIDI data.
