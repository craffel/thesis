This thesis is dedicated to Joanna Percher.
Thanks for being with me every step of the way, and I look forward to our next adventure.
I love you.

I'm lucky to have overall had an extremely positive PhD experience, and I have a lot of people to thank for that.

Incidentally, I will be the last person in my immediate family to receive a PhD, though I overlapped some with my brother and Mom.
Combined with the fact that they have known me my whole life, this gave them an unmatched ability to guide me through this process.
Their support, advice, and love has brought me to where I am.
Mom and Dad, thank you for giving me my inquisitiveness and an occasional one-track mind, and always being available to talk.
Forrest, thanks for checking in so often and for your humor when I needed it.

DAn is a strange advisor in that he only really gives you advice if you ask for it, and he certainly has never insisted that I work on one project or another.
In this sense, his relationship to me more closely resembled that of an interested bystander, who gently provided insight and guidance when necessary.
Nevertheless, looking back at the work I did at LabROSA reveals his strong influence in my ideas and their execution.
In other words, he managed to excel as an advisor while never acting the part, which in the end has gained me both a mentor and a friend.
DAn, thank you for this opportunity.

I'm fortunate to have a thesis committee whose members I can also call friends.
Brian undoubtedly became a sort of ``second advisor'' to me during his years as a postdoc at LabROSA, and much of what I learned during that time and my research philosophy is thanks to him.
John's instruction early in my PhD made me certain that I had chosen the right program.
Nima supported my creation of the Columbia Neural Network Reading Group/Seminar Series, which proved to be an excellent way for me to learn much of the background needed for my thesis work.
Michael provided a sort of spiritual predecessor at LabROSA, and more concretely furnished the thesis template I am using.
Thank you all for being a part of this journey.

I joined LabROSA at the same time as two other PhD students, Zhuo and Dawen.
It's been great to have comrades to share this experience with.
While we seldom worked together directly, we certainly benefitted from each other's company, discussions, and insight.
I hope our paths continue to cross in the future.

Every summer, I hosted interns who each contributed in one way or another to the work in this thesis.
Kitty, Hilary, Dylan, and Rafael, thank you for your help.
LabROSA itself has also hosted many people who I'm happy to have met and worked with over the years.
Matt, Thierry, Courtenay, Rachel, Cyril, H{\'e}l{\`e}ne, Diego, Minshu, James, Andy, and Douglas, thanks for making LabROSA a fun place to do my PhD.
Our proximity to the Music and Audio Research Lab at NYU has also gained me friends and occasional collaborators.
Eric, Justin, Uri, and Rachel, it has been great to get to know you all.

While I don't think I would otherwise have ever moved to New York City, it became my home over the course of three years thanks to my friends there.
Thank you Alec, Annie, Hunter, Emelio, Josh, Sarah, Laura, Emma, and everyone else who I ever had fun with while living in NYC.

The work in this thesis would have been much harder and taken much longer if not for the huge amount of open-source software I used which is distributed free of charge and restrictions.
I particularly want to thank the developers of \texttt{Python}, \texttt{IPython}, \texttt{numpy}, \texttt{scipy}, \texttt{librosa}, \texttt{Theano}, \texttt{lasagne}, \texttt{numba}, \texttt{matplotlib}, \texttt{Spearmint}, \texttt{python-midi}, and \texttt{whoosh}.
I also deeply appreciate the feedback and contributions to my own software projects from users and contributors.

I am grateful to the National Science Foundation for providing the majority of funding for my PhD.

Finally, I'd like to thank my cat Dexter, for his unwavering support.
