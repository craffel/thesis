Sequences of feature vectors are a natural way of representing temporal data.
Given a database of sequences, a fundamental task is to find the database entry which is the most similar to a query.
In this thesis, we present learning-based methods for efficiently and accurately comparing sequences in order to facilitate large-scale sequence search.
Throughout, we will focus on the problem of matching MIDI files (a digital score format) to a large collection of audio recordings of music.
The combination of our proposed approaches enables us to create the largest corpus of paired MIDI files and audio recordings ever assembled.

Dynamic time warping (DTW) has proven to be an extremely effective method for both aligning and matching sequences.
However, its performance is heavily affected by factors such as the feature representation used and its adjustable parameters.
We therefore investigate automatically optimizing DTW-based alignment and matching of MIDI and audio data.
Our approach uses Bayesian optimization to tune system design and parameters over a synthetically-created dataset of audio and MIDI pairs.
We then perform an exhaustive search over DTW score normalization techniques to find the optimal method for reporting a reliable alignment confidence score, as required in matching tasks.
This results in a DTW-based system which is conceptually simple and highly accurate at both alignment and matching.
We also verify that this system achieves high performance in a large-scale qualitative evaluation of real-world alignments.

Unfortunately, DTW can be far too inefficient for large-scale search when sequences are very long and consist of high-dimensional feature vectors.
We therefore propose a method for mapping sequences of continuously-valued feature vectors to downsampled sequences of binary vectors.
Our approach involves training a pair of convolutional networks to map paired groups of subsequent feature vectors to a Hamming space where similarity is preserved.
Evaluated on the task of matching MIDI files to a large database of audio recordings, we show that this technique enables 99.99\% of the database to be discarded with a modest false reject rate while only requiring 0.2\% of the time to compute.

Even when sped-up with a more efficient representation, the quadratic complexity of DTW greatly hinders its feasibility for very large-scale search.
This cost can be avoided by mapping entire sequences to fixed-length vectors in an embedded space where sequence similarity is approximated by Euclidean distance.
To achieve this embedding, we propose a feed-forward attention-based neural network model which can integrate arbitrarily long sequences.
We show that this approach can extremely efficiently prune 90\% of our audio recording database with high confidence.

After developing these approaches, we applied them together to the practical task of matching 178,561 unique MIDI files to the Million Song Dataset.
The resulting ``Lakh MIDI Dataset'' provides a potential bounty of ground truth information for audio content-based music information retrieval.
This can include transcription, meter, lyrics, and high-level musicological features.
The reliability of the resulting annotations depends both on the quality of the transcription and the accuracy of the score-to-audio alignment.
We therefore establish a baseline of reliability for score-derived information for different content-based MIR tasks.
Finally, we discuss potential future uses of our dataset and the learning-based sequence comparison methods we developed.
