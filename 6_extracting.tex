\chapter{Extracting Ground-Truth Information from MIDI Files}
\label{ch:extracting}

\section{MIDI Files}

Where did they come from?
Why are there so many of them?

\subsection{Transcriptions}

Useful for transcrtipion, instrument activation, and score-informed source separation.

\subsection{Temporal Information}

Beats, downbeats (start from first beat), time signature, and tempo changes

\subsection{Key Information}

Key changes (how often are they useful?), as well as pitch class histogram and transition matrix

\section{\texttt{pretty\char`_midi}}

Demo of extracting all of the above things.

\section{How Reliable is MIDI-Derived Ground Truth?}

In order to utilize many of these sources of information, we need to align, using the system of 3dtw.
How accurate is the result?
Temporal information is effected most, so let's test beats.

\subsection{Comparison to Hand-Annotated Data}

Results, with description of failure modes.

\subsection{Improving Alignment as a Proxy for Content-Based MIR}

Ground truth is already valuable; we can make it more reliable by making alignment better.
